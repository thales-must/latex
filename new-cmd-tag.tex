\documentclass{article}
\usepackage[UTF8]{ctex}
\usepackage{xcolor}
\usepackage{mdframed}  % 边框支持

% ===== 定义成对命令示例 =====
% 示例1:您的原始示例
\newcommand{\revisedalgorithm}{\begin{mdframed}[linecolor=red,linewidth=1pt]}
\newcommand{\revisedalgorithmend}{\end{mdframed}}

% 示例2:改进的文本框命令
\newcommand{\textbox}[2][red]{%
  \begin{mdframed}[linecolor=#1]
  \begin{minipage}{\linewidth}
  #2
  \end{minipage}
  \end{mdframed}
}

\title{LaTeX成对命令模式教学}
\author{Thales}
\date{\today}

\begin{document}
\maketitle

\section{什么是成对命令模式?}

成对命令模式是指使用两个独立的 \verb|\newcommand| 来模拟 \verb|begin/end| 环境的功能。

\section{您的示例}

您的代码示例:
\begin{verbatim}
\newcommand{\revisedalgorithm}{\begin{mdframed}[linecolor=red]}
\newcommand{\revisedalgorithmend}{\end{mdframed}}
\end{verbatim}

\section{如何使用?}

\subsection{包裹文本内容}

对于文本内容,最好使用专门的文本框命令:

定义:
\begin{verbatim}
\newcommand{\textbox}[2][red]{%
  \begin{mdframed}[linecolor=#1]
  \begin{minipage}{\linewidth}
  #2
  \end{minipage}
  \end{mdframed}
}
\end{verbatim}

使用:
\begin{verbatim}
\textbox{这是红色边框的文本内容。}
\end{verbatim}

实际效果:
\textbox{这是红色边框的文本内容。}

\subsection{使用可选参数}

\begin{verbatim}
\textbox[blue]{这是蓝色边框的文本内容。}
\end{verbatim}

实际效果:
\textbox[blue]{这是蓝色边框的文本内容。}

\section{您的 revisedalgorithm 命令}

您的 \verb|\revisedalgorithm| 命令更适合包裹算法、表格等环境:

定义:
\begin{verbatim}
\newcommand{\revisedalgorithm}{\begin{mdframed}[linecolor=blue]}
\newcommand{\revisedalgorithmend}{\end{mdframed}}
\end{verbatim}

使用示例(假设有算法环境):
\begin{verbatim}
\revisedalgorithm
\begin{algorithm}
算法内容
\end{algorithm}
\revisedalgorithmend
\end{verbatim}

\section{为什么使用这种模式?}

\subsection{优势}

\begin{enumerate}
  \item \textbf{条件控制灵活}:可以方便地添加条件判断
  \item \textbf{命名自由}:可以自定义有意义的命令名
  \item \textbf{简单直观}:不需要定义完整环境
\end{enumerate}


\end{document}