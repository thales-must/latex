\documentclass{article}
\usepackage[UTF8]{ctex}
\usepackage{xcolor}

\title{LaTeX条件编译:论文修订标记开关}
\author{Thales}
\date{\today}

\begin{document}
\maketitle

\section{条件编译基础}

\subsection{定义条件开关}
\begin{verbatim}
\newif\ifshowmark  % 定义名为 showmark 的条件开关
\end{verbatim}

\subsection{设置开关状态}
\begin{verbatim}
\showmarktrue   % 打开:显示修订标记
%\showmarkfalse % 关闭:隐藏修订标记
\end{verbatim}

\section{实际示例}

\subsection{定义标记命令}
\begin{verbatim}
\newif\ifshowmark
\showmarktrue   % 设为 true 显示标记

\ifshowmark
    % 打开时:文本显示为红色(有标记)
    \newcommand{\marked}[1]{\textcolor{red}{#1}}
\else
    % 关闭时:正常显示文本(无标记)
    \newcommand{\marked}[1]{#1}
\fi
\end{verbatim}

\subsection{正确做法:避免重复定义}

\begin{verbatim}
\newif\ifshowmark
\showmarktrue

% 先统一定义默认版本
\providecommand{\marked}[1]{#1}

% 根据条件重新定义
\ifshowmark
    \renewcommand{\marked}[1]{\textcolor{red}{#1}}
\fi
\end{verbatim}

\section{使用效果演示}

\subsection{当前设置}
\begin{verbatim}
\newif\ifshowmark
\showmarktrue  % 当前打开标记模式

% 统一定义
\providecommand{\marked}[1]{#1}

% 条件重定义
\ifshowmark
    \renewcommand{\marked}[1]{\textcolor{red}{#1}}
\fi
\end{verbatim}

使用示例:
\begin{verbatim}
原始句子:机器学习算法效果很好。
修改后:\marked{深度学习}算法效果\marked{非常好}。
\end{verbatim}

实际效果:
\newif\ifshowmark
\showmarktrue
\providecommand{\marked}[1]{#1}
\ifshowmark
  \renewcommand{\marked}[1]{\textcolor{red}{#1}}
\fi
原始句子:机器学习算法效果很好。
修改后:\marked{深度学习}算法效果\marked{非常好}。

\section{切换模式演示}

\subsection{模式1:显示标记 (showmarktrue)}
\begin{verbatim}
\newif\ifshowmark
\showmarktrue

\providecommand{\marked}[1]{#1}
\ifshowmark
    \renewcommand{\marked}[1]{\textcolor{red}{#1}}
\fi

使用:\marked{修改的内容}
\end{verbatim}

效果:修改的内容会显示为\newif\ifshowmark\showmarktrue\providecommand{\marked}[1]{#1}\ifshowmark\renewcommand{\marked}[1]{\textcolor{red}{#1}}\fi\textcolor{red}{红色}(有标记)

\subsection{模式2:隐藏标记 (showmarkfalse)}
\begin{verbatim}
\newif\ifshowmark
\showmarkfalse

\providecommand{\marked}[1]{#1}
\ifshowmark
    \renewcommand{\marked}[1]{\textcolor{red}{#1}}
\fi

使用:\marked{修改的内容}
\end{verbatim}

效果:修改的内容会\newif\ifshowmark\showmarkfalse\providecommand{\marked}[1]{#1}\ifshowmark\renewcommand{\marked}[1]{\textcolor{red}{#1}}\fi正常显示(无标记)

\section{完整论文示例}

\begin{verbatim}
% === 在论文导言区 ===
\newif\ifshowmark
\showmarktrue  % 写作时打开,提交时关闭

\providecommand{\marked}[1]{#1}
\providecommand{\added}[1]{#1}
\providecommand{\deleted}[1]{}

\ifshowmark
    \renewcommand{\marked}[1]{\textcolor{red}{#1}}
    \renewcommand{\added}[1]{\textcolor{blue}{#1}}
    \renewcommand{\deleted}[1]{\textcolor{red}{\sout{#1}}}
\fi

% === 在正文中 ===
本文提出了一种\added{新的}算法。
实验结果\marked{显著优于}传统方法。
我们\deleted{抛弃了}优化了原有框架。
\end{verbatim}

\section{使用流程}

\begin{enumerate}
  \item 设置 \verb|\showmarktrue|,编译
  \item 得到:得到有标记的PDF
  \item 改为 \verb|\showmarkfalse|,重新编译
  \item 得到:干净的最终版本PDF
\end{enumerate}


\end{document}